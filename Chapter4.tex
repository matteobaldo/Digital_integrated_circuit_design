\chapter{Inverter power consumption}

\section{Dynamic power consumtion}
Only in case of charge of a capacitance so in a LOW to HIGH transition at the output of an inverter. \\
The most general formula for the power dissipated in this process is
\begin{equation}
P_{dyn}=C_LV_{DD}^2f_{1\rightarrow 0}
\end{equation}
where $f_{1\rightarrow 0}$ is the frequency of the 0-1 transition at the output node. This term can be deconposed in 2 factor the clock of the circuit and the switching activity $\alpha_{SW}$ that is the probability to have a transition from 0 to 1 at the output.\\
\vspace{2mm}
\tab In case of a square wave we have 
\begin{equation}
f_{1\rightarrow 0}=f_{clk}\cdot \alpha_{SW}=f_{clk}\cdot \frac{1}{2}
\end{equation}
\tab In case of a randoom signal 
\begin{equation}
f_{1\rightarrow 0}=f_{clk}\cdot \alpha_{SW}=f_{clk}\cdot \frac{1}{4}
\end{equation}
\vspace{5mm}
In the end we can write this 2 final equation for dynamic power dissipation
\begin{equation}
P_{dyn}=C_LV_{DD}^2 f_{clk}\cdot \alpha_{SW}
\end{equation}
\begin{equation}
E_{dyn}=C_LV_{DD}^2\alpha_{SW}
\end{equation}
In case of a chain of inverters $C_L$ is the sum of all the capacitance at the output nodes of the single inverters.\\

\section{Static power consumption}
